
Kohärenz

Die zeitliche Kohärenz bezieht sich auf den Grad der Ähnlichkeit zwischen zwei
Lichtwellen, die von derselben Quelle, aber zu unterschiedlichen Zeiten
ausgesendet werden. Lichtwellen bestehen aus oszillierenden elektrischen und
magnetischen Feldern, und damit es zu einer Interferenz kommt, müssen die
beiden Wellen eine konstante Phasenbeziehung haben. Das bedeutet, dass die
Phasendifferenz zwischen den beiden Wellen über einen längeren Zeitraum
konstant bleiben muss. Ändert sich die Phasenbeziehung schnell, wird von einer
geringen zeitlichen Kohärenz der Wellen ausgegangen. Bleibt die Phasenbeziehung
dagegen über einen langen Zeitraum konstant, so wird von einer hohen zeitlichen
Kohärenz ausgegangen.

Die räumliche Kohärenz hingegen bezieht sich auf den Grad der Ähnlichkeit
zwischen zwei Lichtwellen, die von derselben Quelle, aber an unterschiedlichen
Orten im Raum ausgesendet werden. Damit es zu einer Interferenz kommt, müssen
die beiden Wellen über ihre gesamte Wellenfront eine konstante Phasenbeziehung
aufweisen. Das bedeutet, dass die Phasendifferenz zwischen zwei beliebigen
Punkten der Wellenfronten konstant bleiben sollte. Wenn sich die
Phasenbeziehung über die Wellenfronten hinweg schnell ändert, wird von einer
geringen räumlichen Kohärenz der Wellen ausgegangen. Bleibt die Phasenbeziehung
dagegen über die Wellenfronten hinweg konstant, so wird von einer hohen
räumlichen Kohärenz ausgegangen.

Die Kohärenzlänge ist ein Maß dafür, wie weit sich das Licht unter Beibehaltung
seiner Kohärenz ausbreiten kann. Sie ist die Strecke, über die die
Phasenbeziehung zwischen zwei Punkten auf einer Wellenfront konstant bleibt.
Die Kohärenzlänge verhält sich umgekehrt proportional zur spektralen Bandbreite
der Lichtquelle. Eine schmalbandige Lichtquelle mit einer kleinen spektralen
Bandbreite hat eine große Kohärenzlänge, während eine breitbandige Lichtquelle
mit einer großen spektralen Bandbreite eine kleine Kohärenzlänge hat.

Zur Erzeugung von Interferenzmustern mit Licht ist eine Lichtquelle mit hoher
zeitlicher Kohärenz, hoher räumlicher Kohärenz und einer ausreichend langen
Kohärenzlänge erforderlich. Eine hohe zeitliche Kohärenz gewährleistet, dass
die Phasenbeziehung zwischen zwei Wellen über einen längeren Zeitraum konstant
bleibt, so dass sie konstruktiv oder destruktiv interferieren können. Eine hohe
räumliche Kohärenz stellt sicher, dass die Phasenbeziehung zwischen
verschiedenen Punkten auf den Wellenfronten konstant bleibt, so dass die Wellen
gleichmäßig über das Muster interferieren können. Die Kohärenzlänge bestimmt
die Größe des Bereichs, in dem Interferenz auftreten kann. Eine große
Kohärenzlänge ist notwendig, um gut definierte Interferenzstreifen zu
beobachten, während eine kurze Kohärenzlänge zu verschwommenen oder
verwaschenen Mustern führt.

% Young grundlagen

Das Youngsche Doppelspaltexperiment ist ein klassisches Experiment, das die
Wellennatur des Lichts und das Phänomen der Interferenz demonstriert. Dabei
wird Licht durch zwei eng beieinander liegende Spaltöffnungen geleitet und das
sich ergebende Muster auf einem hinter den Spaltöffnungen angebrachten
Bildschirm beobachtet.

Bei diesem Experiment wird eine kohärente Lichtquelle, z. B. ein Laser,
verwendet, um eine hohe zeitliche und räumliche Kohärenz zu gewährleisten. Das
Licht wird durch zwei schmale Schlitze geleitet, wodurch zwei
Lichtwellenquellen entstehen, die sich als halbkreisförmige Wellenfronten
ausbreiten. Diese Wellenfronten überlagern sich dann und interferieren
miteinander.

Um die Kriterien für konstruktive Interferenz im Young'schen
Doppelspaltexperiment zu verstehen, betrachten wir das Konzept der optischen
Wegdifferenz (OPD). Die OPD ist der Unterschied in der Strecke, die die
Lichtwellen von den beiden Spaltöffnungen bis zu einem bestimmten Punkt auf dem
Bildschirm zurücklegen. Wenn diese Differenz ein ganzzahliges Vielfaches der
Wellenlänge des Lichts ist, tritt an diesem Punkt konstruktive Interferenz auf.

Mathematisch lässt sich die Bedingung für konstruktive Interferenz wie folgt
ausdrücken:

% gleichung konstruktive Interferenzbedingung

wobei d der Abstand zwischen den beiden Schlitzen und θ der Winkel zwischen der
Verbindungslinie zwischen dem Punkt auf dem Schirm und den Schlitzen und der
Senkrechten zum Schirm ist.

Wenn die OPD gleich mλ ist, wobei m eine ganze Zahl ist, die die Ordnung des
Interferenzstreifens angibt, und λ die Wellenlänge des Lichts ist, tritt
konstruktive Interferenz auf. Das bedeutet, dass die Wellen der beiden Schlitze
an diesem Punkt des Schirms in Phase eintreffen, was zu einem hellen Streifen
führt.

Die quadrierte Amplitude der Welle steht in direktem Zusammenhang mit der
Lichtintensität, die an einem bestimmten Punkt des Bildschirms beobachtet wird.
Durch Quadrieren der Amplitude erhält man also das Interferenzmuster des
Doppelspalts. Das Muster besteht aus abwechselnd hellen und dunklen Streifen,
die als Interferenzstreifen oder -bänder bezeichnet werden.

% gleichung Doppelspalte

Zusätzlich zum Interferenzmuster des Doppelspalts wird jedoch noch ein weiteres
Interferenzmuster überlagert. Dieses zusätzliche Muster entsteht durch die
Interferenz von Lichtwellen, die durch jeden einzelnen Spalt laufen und sich
dann beugen, wodurch Einzelspalt-Interferenzmuster entstehen. Die einzelnen
Schlitze sind keine punktförmigen Quellen, und die gebeugten Wellen jedes
Spalts interferieren mit sich selbst.

Das Interferenzmuster der Einzelspalte ist durch ein zentrales Maximum und eine
Reihe kleinerer Maxima und Minima auf beiden Seiten gekennzeichnet. Dieses
Muster wird mit dem Interferenzmuster der Doppelspalte überlagert, wodurch sich
ein komplexeres Gesamtmuster ergibt, das auf dem Bildschirm zu sehen ist. Das
kombinierte Muster weist sowohl die Interferenzstreifen des Doppelspalts als
auch das Interferenzmuster des Einzelspalts auf.

% gleichung einzelspalt

Das im Young'schen Doppelspaltexperiment beobachtete Muster ist also die
Überlagerung von zwei Interferenzmustern: den Interferenzstreifen, die von den
Doppelspalten erzeugt werden, und dem Interferenzmuster, das sich aus der
Beugung der Einzelspalte ergibt. Dieses Experiment ist ein starker Beweis für
die Wellennatur des Lichts und das Phänomen der Interferenz.

% Malus gesetzt bitte verkuerzen
Das Gesetzt von Malus-Gesetz beschreibt die Beziehung zwischen der Intensität
des polarisierten Lichts, das durch einen Analysator übertragen wird, und dem
Winkel zwischen der Polarisationsrichtung des einfallenden Lichts und der
Übertragungsachse des Analysators.

Nach dem Malus-Gesetz ist die Intensität (I) des durchgelassenen Lichts gegeben
durch:

I = I₀ * cos²(θ)

wobei I₀ die anfängliche Intensität des einfallenden polarisierten Lichts und θ
der Winkel zwischen der Polarisationsrichtung des einfallenden Lichts und der
Transmissionsachse des Analysators ist.

Das Malus-Gesetz beruht auf dem Prinzip der Polarisation. Wenn unpolarisiertes
Licht einen Polarisator durchläuft, wird es polarisiert und seine elektrischen
Feldschwingungen auf eine bestimmte Richtung beschränkt. Das polarisierte Licht
ist durch seine Polarisationsrichtung gekennzeichnet, die senkrecht zur
Transmissionsachse des Polarisators verläuft.

Wenn das polarisierte Licht einen Analysator durchläuft, der ein weiterer
Polarisator mit einer bestimmten Transmissionsachse ist, hängt die Intensität
des übertragenen Lichts von der relativen Ausrichtung zwischen der
Polarisationsrichtung des einfallenden Lichts und der Transmissionsachse des
Analysators ab.

Wenn die Polarisationsrichtung des einfallenden Lichts perfekt mit der
Transmissionsachse des Analysators ausgerichtet ist (θ = 0), ist die
transmittierte Intensität (I) maximal und entspricht der Ausgangsintensität
(I₀).

Wenn der Winkel (θ) zwischen der Polarisationsrichtung des einfallenden Lichts
und der Transmissionsachse des Analysators zunimmt, nimmt die transmittierte
Intensität ab. Bei θ = 90 Grad ist die durchgelassene Intensität minimal und
wird zu Null. Dies ist der Fall, wenn die Polarisationsrichtung des
einfallenden Lichts senkrecht zur Transmissionsachse des Analysators steht.

Mathematisch gesehen zeigt das Malus-Gesetz, dass die übertragene Intensität
proportional zum Quadrat des Kosinus des Winkels zwischen der
Polarisationsrichtung und der Transmissionsachse ist. Mit zunehmendem Winkel
nimmt der Kosinus des Winkels ab, was zu einer Abnahme der übertragenen
Intensität führt.
